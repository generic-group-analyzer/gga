\documentclass[12pt]{book}
\usepackage[latin1]{inputenc}
\usepackage[headings]{fullpage}
\usepackage[novisiblespaces]{ocamlweb}
\pagestyle{headings}
\usepackage{hyperref,framed,amsmath,amsfonts,amsthm}

\begin{document}

\renewcommand{\ocwmodule}[1]{\newpage \section{Module #1}}
\renewcommand{\ocwyaccmodule}[1]{\newpage \section{Module #1 (Yacc)}}
\renewcommand{\ocwlexmodule}[1]{\newpage \section{Module #1 (Lex)}}
\renewcommand{\ocwinterface}[1]{\newpage \section{Interface #1}}

\newcommand{\ic}[1]{\vspace{-1.75em}\begin{leftbar}\noindent #1\end{leftbar}\vspace{-0.75em}}


\newcommand{\hd}[1]{\vspace{-1.75em}\begin{framed}\noindent\bf #1\end{framed}\vspace{-0.75em}}


\title{Generic Group Tool (ggt)}
\author{Benedikt Schmidt \& Edvard Fagerholm}

\maketitle
 
\tableofcontents
\newpage

\section*{Introduction}

This text contains all relevant code from the implementation.
For clarity, we hide module opens, module abbreviations, and
all pretty-printing code. Please refer to the source code for
these.