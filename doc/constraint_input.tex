
\newcommand{\range}[1]{r_{#1} \in [\alpha_{#1},\beta_{#1}]}%
\renewcommand{\brack}[1]{[#1]}%
%
\label{assumption_def}%
For a a symmetric leveled $k$-linear map,
  random variables $\vec{X}$ and range limits $\vec{l}$,
  the adversary input is of the following form
\begin{align*}
  I_1 :={}& \forall \range{1,1},\ldots,\range{1,w_1}:\, \vec{X}^{\vec{f_{1}}} & \text{ in group }\lambda_1\\
  \ldots & \\
  I_n :={}& \forall \range{n,1},\ldots,\range{n,w_n}:\, \vec{X}^{\vec{f_{1}}} & \text{ in group }\lambda_n
\end{align*}
  and the challenge is of the form
  $C := \vec{X}^{\vec{g}} \text{ in group }\lambda$\footnote{
    Everything generalizes to polynomials $C$ since
    the method can be applied to each monomial in $C$.}.
Here, we assume that $g_i \in \mathbb{Z}[k,\vec{l}]$,
  $f_{j,i} \in \mathbb{Z}[k,\vec{l},r_{j,1},\ldots,r_{j,w_j}]$,
  $\alpha_{j,i} \in \mathbb{Z}[\vec{l}]$,
  $\beta_{j,i} \in \mathbb{Z}[\vec{l}]$, and
  $\lambda_i$ is either a positive integer or
  of the form $k - i$ for a positive integer $i$.
For decisional problems, it must hold that $\lambda=k$.

Note that we always assume that $\alpha_i < \beta_i$ since we assume that
  all ranges are non-empty and increasing.