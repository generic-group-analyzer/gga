%
%
Let the assumption with inputs $I_i$ and challenge $C$ be
  defined as in Chapter~\ref{assumption_def}.
Since we require $\lambda = k$ for decisional problems, the adversary
  cannot win unless he computes $C$ from $I_1,\ldots,I_n$ for both kinds of problems.
It is not hard to see that addition is useless to compute a monomial from monomials
  and we only have to consider multiplications.

More precisely, the adversary can win iff there are $\delta_i \in \mathbb{N}$ for $i \in [n]$
  such that holds that\footnote{
    Defining $S_1 * S_2 := \{ s_1 * s_2 \mid s_1 \in S_1 \land s_2 \in S_2\}$
    and $S^\delta = \{ \Pi_{i=1}^{\delta} s_i \mid s_1 \in S \land \ldots \land s_{\delta} \in S\}$
    as usual.
  }
$
  C \in (I_1^{\delta_1} * \cdots * I_n^{\delta_n})
$
  and the group setting allows for the computation of the product on the right-hand-side, i.e., 
$
  \Sigma_{i=1}^n\, \delta_i * \lambda_i = \lambda \text{.}
$
%
It is therefore sufficient to perform the following tasks to analyze such assumptions:
\begin{enumerate}
\item Compute a range expression $J$ that characterizes the set $I_1 * I_2$.
\item Check if $C \in I$ for a a monomial $C$ and a range expression $I$.
\item Compute a range expression $J$ that characterizes the set $I^\delta$ for a variable $\delta$.
\end{enumerate}
We will now describe our approach to handle these tasks.

\paragraph{1.}%
We can rename all range indices apart and then perform the following computation
$$
  (\forall \vec{r} \in \vec{R}:\, \vec{X}^{\vec{f}})
  *  (\forall \vec{r'} \in \vec{R'}:\, \vec{X}^{\vec{h}})
=
    (\forall \vec{r} \in \vec{R}, \vec{r'} \in \vec{R'}:\,
         \vec{X}^{\vec{f} + \vec{h}}) \text{.}
$$

\paragraph{2.}%
Let $C := \vec{X}^{\vec{g}}$ and
$I := \forall \range{1}, \ldots, \range{w}:\, \vec{X}^{\vec{f}}$.
We can then define a translation into the following system of polynomial constraints:
\begin{align}
 \alpha_i \leq \beta_i \\
 0 \leq{}& \delta_i \\
 \Sigma_{i=1}^n\, \delta_i * \lambda_i ={}& \lambda \\
 \alpha_j \leq r_j \leq{}& \beta_j \\
 k >{}& i \quad \text{for all levels $k - i$} \\
 f_i ={}& g_i
\end{align}
The constraint system is over the integer variables $k$, $\vec{l}$,
  $r_{j}$, and $\delta_i$.
%Using Z3 we can prove unsatisfiability or find models.
%Note that there are classes of inputs for which
%  the satisfiability problem is decidable.
%\begin{framed}
%  \noindent {\bf FIXME:} We probably have to allow for user-defined constraints
%  (such as $k > 2$) in the problem definition. We might also have to check
%  well-formedness of the problem, e.g., exponents of $X_i$ always positive.
%\end{framed}

\paragraph{3.}%
Let $I$ be defined as above.
If $f$ is linear in $\vec{r}$ (considering $k$ and $\vec{l}$
  as constants), then we can write $\vec{f}(k,\vec{l},\vec{r})$
  as $r_1*\phi_1(k,\vec{l}) + \ldots + r_k*\phi_k(k,\vec{l}) + \psi(k,\vec{l})$.
In this case, $I^{\delta}$ is characterized by the following range expression:
$$
    \forall r_1 \in \brack{\delta\alpha_1, \delta\beta_1},
            \ldots,
            r_k \in \brack{\delta\alpha_k, \delta\beta_k}:
            \vec{X}^{r_1*\vec{\phi}_1(k,\vec{l}) + \ldots + r_w*\vec{\phi}_w(k,\vec{l})
                     + \delta*\vec{\psi}(k,\vec{l})} \text{.}
$$

\begin{proof}
We prove that for all monomials $D$, the following holds:
\begin{align*}
   & D \in I^{\delta} \\
%
{}\Leftrightarrow{} &
  \exists \vec{r}_{1,1..w},\ldots,\vec{r}_{\delta,1..w} \in [\vec{\alpha}, \vec{\beta}]:\,
    D = \vec{X}^{\vec{f}(k,\vec{l},\vec{r}_{1,1..w}) + \ldots + \vec{f}(k,\vec{l},\vec{r}_{\delta,1..w}) } \\
%
{}\Leftrightarrow{} &
  \exists \vec{r}_{1,1..w},\ldots,\vec{r}_{\delta,1..w} \in [\vec{\alpha}, \vec{\beta}]:\,
    D = \vec{X}^{ \Sigma_{i=1}^{\delta} (r_{i,1}*\vec{\phi_1}(k,\vec{l}) + \ldots + r_{i,w}*\vec{\phi_w}(k,\vec{l})
                  + \vec{\psi}(k,\vec{l}))
                } \\
{}\Leftrightarrow{} &
  \exists \vec{r}_{1,1..w},\ldots,\vec{r}_{\delta,1..w} \in [\vec{\alpha}, \vec{\beta}]:\,
    D = \vec{X}^{   (\Sigma_{i=1}^{\delta} r_{i,1})*\vec{\phi_1}(k,\vec{l}) + \ldots
                  + (\Sigma_{i=1}^{\delta} r_{i,w})*\vec{\phi_w}(k,\vec{l})
                  + \delta*\vec{\psi}(k,\vec{l})
                } \\
{}\Leftrightarrow{} &
  \exists \vec{r}_{1..w} \in [\delta * \vec{\alpha}, \delta * \vec{\beta}]:\,
    D = \vec{X}^{   r_{1}*\vec{\phi_1}(k,\vec{l}) + \ldots
                  + r_{w}*\vec{\phi_w}(k,\vec{l})
                  + \delta*\vec{\psi}(k,\vec{l})
                } \\
{}\Leftrightarrow{} &
  D \in \forall r_1 \in \delta[\alpha_1, \beta_1], \ldots, r_w \in \delta[\alpha_w, \beta_w]:\,
    \vec{X}^{r_{1}*\vec{\phi_1}(k,\vec{l}) + \ldots + r_{w}*\vec{\phi_w}(k,\vec{l}) + \delta*\vec{\psi}(k,\vec{l})}
%
\end{align*}
In the second to last step, we exploit that
\[
   \{ \Sigma_{i=1}^{\delta} r_{i,j} | \mid r_{1,j}, \ldots, r_{\delta,j} \in [\alpha_j, \beta_j] \}
   = [\delta*\alpha_j, \delta*\beta_j]\text{.}
\]
Indeed, if $\alpha_j \leq \beta_j$, then the minimum value in the set on the left-hand-side
  is $\delta*\alpha_j$, the maximim value is $\delta*\beta_j$, and set includes all values
  in between.
If $\alpha_j > \beta_j$, then the same argument applies since $[\alpha_j,\beta_j] = [\beta_j,\alpha_j]$
  as sets.  
\end{proof}

