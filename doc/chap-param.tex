\chapter{Parametric Assumptions}

This chapter covers the implementation of our approach for
analyzing parametric assumptions.
We support all assumptions satisfying the following restrictions:
\begin{enumerate}
\item The group setting models a symmetric leveled $k$-linear map.
\item The problem is a computational problem or a decisional
 problem. Decisional problems must be real-or-random and
 the challenge must be in the target group.
\item The input given to the adversary must be a monomial.
\item The challenge must be a monomial.\footnote{
    Note that everything generalizes to polynomials $C$ since
    the method can be applied to each monomial in $C$.}
\end{enumerate}%
%
\newcommand{\range}[1]{r_{#1} \in [\alpha_{#1},\beta_{#1}]}%
\renewcommand{\brack}[1]{[#1]}%
%
\label{assumption_def}%
For a a symmetric leveled $k$-linear map,
  random variables $\vec{X}$ and range limits $\vec{l}$,
  the adversary input is of the following form
\begin{align*}
  I_1 :={}& \forall \range{1,1},\ldots,\range{1,w_1}:\, \vec{X}^{\vec{f_{1}}}
    & \text{ in group }\lambda_1\\
  \ldots & \\
  I_n :={}& \forall \range{n,1},\ldots,\range{n,w_n}:\, \vec{X}^{\vec{f_{n}}}
    & \text{ in group }\lambda_n
\end{align*}
  and the challenge is of the form
  $C := \vec{X}^{\vec{g}} \text{ in group }\lambda$.
We assume the following:
\begin{description}
\item[WF1] $f_{j,i} \in \mathbb{Z}[k,\vec{l},r_{j,1},\ldots,r_{j,w_j}]$,
          no overshadowing of range indices, and
          $\alpha_{j,i},\beta_{j,i} \in \mathbb{Z}[\vec{l}]$
\item[WF2] $g_i \in \mathbb{Z}[k,\vec{l}]$
\item[WF3] $\lambda_i$ is either a positive integer or
  of the form $k - i$ for a positive integer $i$
\item[WF4] $\lambda=k$ for decisional problems
\end{description}
%
Note that we always assume that $\alpha_i \leq \beta_i$ since
  we assume that all ranges are non-empty and increasing.
We also assume that $k > i$ for all such levels occuring in the
  assumption.