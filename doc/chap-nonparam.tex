\chapter{Non-parametric Assumptions}

This chapter covers the implementation of our approach for
analyzing non-parametric assumptions.
In the following, we identify elements $[f]P_i \in \mathbb{G}_i$ with
  their exponent-polynomials~$f$.

An assumption is defined by giving the group setting and
  specifying if the problem is \emph{decisional} or \emph{computational}.
In the decisional case, the user specifies a right-list and a left-list of
  adversary inputs.
If these lists differ only in the last element, we call the problem
  a \emph{real-or-random} problem.
In the computational case, the user specifies a list of adversary inputs
  and the challenge.