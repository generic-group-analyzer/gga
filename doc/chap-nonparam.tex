\chapter{Non-parametric Assumptions}

This chapter covers the implementation of our approach for
analyzing non-parametric assumptions.
In the following, we identify elements $[f]P_i \in \mathbb{G}_i$ with
  their exponent-polynomials~$f$.

An assumption is defined by giving the group setting and
  specifying if the problem is decisional or computational.
In the decisional case, the user specifies a right-list and a left-list of
  adversary inputs.
If these lists differ only in the last element, we call the problem
  a \emph{real-or-random} problem.
In the computational case, the user specifies a list of adversary inputs
  and the challenge.

\begin{verbatim}
class Grouptype:
  def __init__(self,gt):
    self.isoms    = [ (a['fst'], a['snd']) for a in gt['isoms']]
    self.maps     = [ (sorted(a['fst']), a['snd']) for a in gt['maps']]
    self.leveled_groupnames = infer_levels( gt['groupnames']
                                          , self.isoms
                                          , self.maps)
\end{verbatim}

\begin{verbatim}
class Problem:
  def __init__(self, prob):
    self.grouptype           = Grouptype(prob['grouptype'])
    self.is_decisional       = prob['isDecisional']
    self.decisional_left     = prob['decisional_left']
    self.decisional_right    = prob['decisional_right']
    self.computational_input = prob['computational_input']
    if prob['computational_compute'] is not None:
      self.computational_compute_group = prob['computational_compute']['fst']
      self.computational_compute_poly  = prob['computational_compute']['snd']
\end{verbatim}