\chapter{Interactive Assumptions}

This chapter covers the implementation of our approach for
analyzing interactive assumptions.
%
We get an input of the following form:
\begin{verbatim}
emaps G1 * G2 -> GT.
isom G1 -> G2.

input [ X, Y ] in G1.

oracle O(m : Fq) = sample A:G1, (A, A*Y, A*X + m*A*X*Y).

win(U:G1, V:G1, W:G1, mm) = U <> 0 /\ mm <> m_i /\ V = UX /\ W = U*X + m*U*X*Y.
\end{verbatim}
%
For now, we make the following assumptions:
\begin{enumerate}
\item Either the group setting is a generic group or the input, oracle
  arguments and return values, and winning condition input
  are all in one group. In the last case, we exploit that the
  problem is computational.
\item All oracle inputs are of type \verb!Fq!. Allowing for group
  elements complicates the definition.
\end{enumerate}
%
We first compute a formal sum for each \verb!win! input of type $\group$
  as follows:
\begin{enumerate}
\item Assume that the adversary is given inputs
  $\vec{f}$ where $f_j$ defines an element in
  $\group$ over random variables~$\vec{X}$.
\item That there is one oracle taking field elements
  $\vec{m}$ and returning $\vec{g}$
  where $g_j$ defines an element in $\group$
  over the variables $\vec{X}$ and the variables
  $\vec{A}$ sampled in the oracle call.
\item We assume there are $q$ oracle queries.
\item As a first step, we introduce indexed parameters
  $m_{1,j},\ldots,m_{l,j}$ ($j \in [q]$) and
  indexed random variables
  $A_{1,j},\ldots,A_{r,j}$ ($j \in [q]$).
\item Then all computable elements can
  be expressed as linear combinations as follows:
  \[
    \alpha_1 f_1 + \ldots + \alpha_k f_k
    + \Sigma_{i=1}^q \beta_{1,i}\, g_1(\vec{m_i},\vec{A_i},\vec{X})
    \ldots
    + \Sigma_{i=1}^q \beta_{n,i}\, g_n(\vec{m_i},\vec{A_i},\vec{X})
  \]
\item We assume the winning condition takes
  elements $\vec{U}$ in $\group$.
  Then we use $\alpha_i^{(j)}$ and $\beta_i^{(j)}$ (this is a vector) for 
  the coefficients of $U_j$.
\end{enumerate}

We represent such linear combinations as formal sums.